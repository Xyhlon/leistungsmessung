%! TeX program = lualatex
%---------------------------ALLGEMEINE IMPORTS-------------------------------------
\documentclass[12pt,english,ngerman]{scrartcl}
\input{./protokoll_template/template.latex/input/shared_preamble.tex}

% Kopfzeile
\ihead{WS22\\
	09.12.2022} \chead{\textsc{Stark} Matthias - 12004907 \\
	\textsc{Philipp} Maximilian - 11839611}
\ohead{FLAB 1 \\
	Leistungsmessung}
% Fußzeile
\addbibresource{leistungsmessung.bib}

\begin{document}
%todo deckblatt
%\includepdf{./deckblatt3.pdf}
\tableofcontents
\newpage

\section{Aufgabenstellung\label{Auf}}

\begin{itemize}
	\item Leistungsmessung einer ohmschen Last in einem Wechselstromkreis
	\item Wirkleistungsmessung im Drehstromnetz bei einer symmetrischen ohmschen Last in Stern- und Dreieckschaltung mit Aronschaltung
	\item Wirk- und Blindleistungsmessung bei einer allgemeinen Last im Dreiphasennetz
	\item Bauen eines rudimentärern Asynchron-Drehstrommotors
\end{itemize}

\section{Grundlagen}\label{Grund}


\section{Versuchsanordnung}
\label{sec:versuchsanordnung}

\subsection{ohmsche Last in Wechselstromkreis}


\subsection{Symmetrische Last in Dreieckschaltung}


\subsection{Symmetrische Last in Sternschaltung}


\subsection{Asymmetrische Last in Sternschaltung}


\subsection{Asymmetrische Last in Sternschaltung und simulierten Kabelbruch}


\subsection{Wirkleistungsmessung}


\subsection{Blindleistungsmessung}


\subsection{Bau eines rudimentärern Asynchron-Drehstrommotors}





\section{Geräteliste}
\label{sec:geraeteliste}





\section{Versuchsdurchführung und Messergebnisse}
\label{sec:versuchsdurchfuehrung_messergebnisse}

\subsection{ohmsche Last in Wechselstromkreis}


\subsection{Symmetrische Last in Dreieckschaltung}


\subsection{Symmetrische Last in Sternschaltung}


\subsection{Asymmetrische Last in Sternschaltung}


\subsection{Asymmetrische Last in Sternschaltung und simulierten Kabelbruch}


\subsection{Wirkleistungsmessung}


\subsection{Blindleistungsmessung}


\subsection{Bau eines rudimentärern Asynchron-Drehstrommotors}


\section{Auswertung}
\label{sec:auswertung}

\subsection{ohmsche Last in Wechselstromkreis}


\subsection{Symmetrische Last in Dreieckschaltung}


\subsection{Symmetrische Last in Sternschaltung}


\subsection{Asymmetrische Last in Sternschaltung}


\subsection{Asymmetrische Last in Sternschaltung und simulierten Kabelbruch}


\subsection{Wirkleistungsmessung}


\subsection{Blindleistungsmessung}


\subsection{Bau eines rudimentärern Asynchron-Drehstrommotors}


\section{Diskussion}
\label{sec:diskussion}

\subsection{ohmsche Last in Wechselstromkreis}


\subsection{Symmetrische Last in Dreieckschaltung}


\subsection{Symmetrische Last in Sternschaltung}


\subsection{Asymmetrische Last in Sternschaltung}


\subsection{Asymmetrische Last in Sternschaltung und simulierten Kabelbruch}


\subsection{Wirkleistungsmessung}


\subsection{Blindleistungsmessung}


\subsection{Bau eines rudimentärern Asynchron-Drehstrommotors}

\section{Zusammenfassung}
\label{sec:zusammenfassung}

\subsection{ohmsche Last in Wechselstromkreis}


\subsection{Symmetrische Last in Dreieckschaltung}


\subsection{Symmetrische Last in Sternschaltung}


\subsection{Asymmetrische Last in Sternschaltung}


\subsection{Asymmetrische Last in Sternschaltung und simulierten Kabelbruch}


\subsection{Wirkleistungsmessung}


\subsection{Blindleistungsmessung}


\subsection{Bau eines rudimentärern Asynchron-Drehstrommotors}

\newpage

\printbibliography
\listoffigures
\listoftables
\end{document}