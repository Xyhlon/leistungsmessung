%! TeX program = lualatex
%---------------------------ALLGEMEINE IMPORTS-------------------------------------
\documentclass[12pt,english,ngerman]{scrartcl}
\input{./protokoll_template/template.latex/input/shared_preamble.tex}

% Kopfzeile
\ihead{WS22\\
	09.12.2022} \chead{\textsc{Stark} Matthias - 12004907 \\
	\textsc{Philipp} Maximilian - 11839611}
\ohead{FLAB 1 \\
	Leistungsmessung}
% Fußzeile
\addbibresource{leistungsmessung.bib}

\begin{document}
%todo deckblatt
%\includepdf{./deckblatt3.pdf}
\tableofcontents
\newpage

\section{Aufgabenstellung\label{Auf}}

\begin{itemize}
	\item Leistungsmessung einer ohmschen Last in einem Wechselstromkreis
	\item Wirkleistungsmessung im Drehstromnetz bei einer symmetrischen ohmschen Last in Stern- und Dreieckschaltung mit Aronschaltung
	\item Wirk- und Blindleistungsmessung bei einer allgemeinen Last im Dreiphasennetz
	\item Bauen eines rudimentärern Asynchron-Drehstrommotors
\end{itemize}

\section{Grundlagen}\label{Grund}


\section{Versuchsanordnung}
\label{sec:versuchsanordnung}

\subsection{ohmsche Last in Wechselstromkreis}

Um die ohmsche Last einer Glühlampe im Wechselstromkreis zu messen, wird folgender Versuchsaufbau aus \autoref{fig:aufbau_ohm} realisiert.

\begin{figure}[H]
	\begin{center}
		\includegraphics[width = 0.5\textwidth]{./figures/aufbau_ohm.png}
	\end{center}
	\caption[Realer Versuchsaufbau für die Messung einer ohmschen Last]
	{Realer Versuchsaufbau für die Messung einer ohmschen Last
	1 \(\dots\) Transformator  \\
	2 \(\dots\) seriell geschaltetes Strommessgerät  \\
	3 \(\dots\) seriell geschaltetes Leistungsmessgerät mit parallelen Anschluss zum Verbraucher  \\
	4 \(\dots\) ohmscher Verbraucher (Glühlampe)  \\
	5 \(\dots\) parallel geschaltetes Spannungsmessgerät
	}\label{fig:aufbau_ohm}
\end{figure}


\subsection{Symmetrische Last in Dreieckschaltung}

Um die Wirkleistung von symmetrischen Verbrauchern in einer Dreiecksschaltung zu Messen, wird eine Aronschaltung
 nach folgendem Schaltplan aus \autoref{fig:aufbau1} realisiert. 
 Der tatsächliche Versuchsaufbau ist in \autoref{fig:aufbau1_echt} ersichtlich.

 \begin{figure}[H]
	\begin{center}
		\includegraphics[width = 0.8\textwidth]{./figures/aufbau1.png}
	\end{center}
	\caption[Schaltplan für die Messung der Wirkleistung mit Aronschaltung für symmetrische
	Verbraucher in Dreiecksschaltung]
	{Schaltplan für die Messung der Wirkleistung mit Aronschaltung für symmetrische
	Verbraucher in Dreiecksschaltung~\cite[]{leistungsmessungvorbereitung}  \\
	$I_i$ \(\dots\) entsprechende Ströme gemessen mit entsprechenden Amperemeter A  \\
	$U_i$ \(\dots\) entsprechende Spannungen gemessen mit entsprechenden Voltmeter V  \\
	$R_i$ \(\dots\) entsprechender Widerstand durch die jeweiligen Verbraucher  \\
	$P_i$ \(\dots\) Powermeter in Aronschaltung
	}\label{fig:aufbau1}
\end{figure}

\begin{figure}[H]
	\begin{center}
		\includegraphics[width = \textwidth]{./figures/aufbau1_echt.png}
	\end{center}
	\caption[Realer Versuchsaufbau für die Messung der Wirkleistung mit Aronschaltung für symmetrische
	Verbraucher in Dreiecksschaltung]
	{Realer Versuchsaufbau für die Messung der Wirkleistung mit Aronschaltung für symmetrische
	Verbraucher in Dreiecksschaltung. (Bei den Kabeln wurde ein Farbschema eingehalten, um eine bessere Übersicht zu ermöglichen.) 
	1 \(\dots\) Versorgungsspannung ($L_1$ rot, $L_2$ blau, $L_3$ gelb) \\
	2 \(\dots\) seriell geschaltete Strommessgeräte  \\
	3 \(\dots\) seriell geschaltete Leistungsmessgeräte mit parallelen Anschlüssen nach der Aronschaltung (grün)\\
	4 \(\dots\) parallel geschaltete Spannungsmessgeräte über die entsprechenden Verbraucher (schwarz)\\
	5 \(\dots\) symmetrisch verteilte ohmsche Verbraucher (Glühlampen)}
	\label{fig:aufbau1_echt}
\end{figure}

\subsection{Symmetrische Last in Sternschaltung}

Um die Wirkleistung von symmetrischen Verbrauchern in einer Sternschaltung zu Messen, wird eine Aronschaltung
nach folgendem Schaltplan aus \autoref{fig:aufbau2} realisiert. 
Der tatsächliche Versuchsaufbau ist in \autoref{fig:aufbau2_echt} ersichtlich.

\begin{figure}[H]
	\begin{center}
		\includegraphics[width = 0.8\textwidth]{./figures/aufbau2.png}
	\end{center}
	\caption[Schaltplan für die Messung der Wirkleistung mit Aronschaltung für symmetrische
	Verbraucher in Sternschaltung]
	{Schaltplan für die Messung der Wirkleistung mit Aronschaltung für symmetrische
	Verbraucher in Sternschaltung~\cite[]{leistungsmessungvorbereitung}  \\
	$I_i$ \(\dots\) entsprechende Ströme gemessen mit entsprechenden Amperemeter A  \\
	$U_i$ \(\dots\) entsprechende Spannungen gemessen mit entsprechenden Voltmeter V  \\
	$R_i$ \(\dots\) entsprechender Widerstand durch die jeweiligen Verbraucher  \\
	$P_i$ \(\dots\) Powermeter in Aronschaltung}
	\label{fig:aufbau2}
\end{figure}

\begin{figure}[H]
	\begin{center}
		\includegraphics[width = \textwidth]{./figures/aufbau2_echt.png}
	\end{center}
	\caption[Realer Versuchsaufbau für die Messung der Wirkleistung mit Aronschaltung für symmetrische
	Verbraucher in Sternschaltung]
	{Realer Versuchsaufbau für die Messung der Wirkleistung mit Aronschaltung für symmetrische
	Verbraucher in Sternschaltung. (Bei den Kabeln wurde ein Farbschema eingehalten, um eine bessere Übersicht zu ermöglichen.) 
	1 \(\dots\) Versorgungsspannung ($L_1$ rot, $L_2$ blau, $L_3$ gelb) \\
	2 \(\dots\) seriell geschaltete Strommessgeräte  \\
	3 \(\dots\) seriell geschaltete Leistungsmessgeräte mit parallelen Anschlüssen nach der Aronschaltung (grün)\\
	4 \(\dots\) parallel geschaltete Spannungsmessgeräte über die entsprechenden Verbraucher (schwarz)\\
	5 \(\dots\) symmetrisch verteilte ohmsche Verbraucher (Glühlampen) \\
	6 \(\dots\) Strommessgerät zwischen Sternpunkt und Neutralleiter (grau)}\label{fig:aufbau2_echt}
\end{figure}


\subsection{Asymmetrische Last in Sternschaltung}

Um eine asymmetrische Last zu erreichen, wird der Aufbau aus \autoref{fig:aufbau2} herangezogen, mit dem Unterschied, 
dass die Glühlampen nicht gleichmäßig auf die Leiter aufgeteilt werden. Die gewählte Konfiguration ist in \autoref{fig:lampenasym} ersichtlich.

\begin{figure}[H]
	\begin{center}
		\includegraphics[width = 0.5\textwidth]{./figures/lampen.png}
	\end{center}
	\caption[Entsprechende Konfiguration für eine asymmetrische Verteilung der Last]
	{Entsprechende Konfiguration für eine asymmetrische Verteilung der Last mit folgenden Verteilungen auf den Strängen: \\
	$L_1$ \(\dots\) 1 x \SI[]{60}{\watt} \\
	$L_2$ \(\dots\) 2 x \SI[]{75}{\watt} \\
	$L_3$ \(\dots\) 1 x \SI[]{75}{\watt} und 2 x \SI[]{60}{\watt}
	}\label{fig:lampenasym}
\end{figure}

\subsection{Asymmetrische Last in Sternschaltung und simulierten Kabelbruch}


Um einen Kabelbruch zu simulieren, wird der Aufbau aus \autoref{fig:aufbau2} herangezogen. Nun wird der Kontakt 
des Neutralleiters unterbrochen, indem das graue Kabel, sichtbar in \autoref{fig:aufbau1_echt}, aus dem 
Strompfad des Multimeters entfernt und in den Spannungsbereich gesteckt wird, um eine Spannungsmessung zu ermöglichen.



\subsection{Wirkleistungsmessung}

Um die Wirkleistung von allgemeinen Verbrauchern in Sternschaltung zu bestimmen, wird die Schaltung
nach folgendem Schaltplan aus \autoref{fig:aufbau3} aufgebaut. 
Der tatsächliche Versuchsaufbau ist in \autoref{fig:aufbau3_echt} ersichtlich.

\begin{figure}[H]
	\begin{center}
		\includegraphics[width = 0.8\textwidth]{./figures/aufbau3.png}
	\end{center}
	\caption[Schaltplan für die Messung der Wirkleistung für allgemeine
	Verbraucher in Sternschaltung]
	{Schaltplan für die Messung der Wirkleistung für allgemeine
	Verbraucher in Sternschaltung~\cite[]{leistungsmessungvorbereitung}  \\
	$I_i$ \(\dots\) entsprechende Ströme gemessen mit entsprechenden Amperemeter A  \\
	$U_i$ \(\dots\) entsprechende Spannungen gemessen mit entsprechenden Voltmeter V  \\
	$R_i$ \(\dots\) entsprechender Widerstand durch die jeweiligen Verbraucher  \\
	$P_i$ \(\dots\) Powermeter}
	\label{fig:aufbau3}
\end{figure}

\begin{figure}[H]
	\begin{center}
		\includegraphics[width = \textwidth]{./figures/aufbau3_echt.png}
	\end{center}
	\caption[Realer Versuchsaufbau für die Messung der Wirkleistung für allgemeine
	Verbraucher in Sternschaltung]
	{Realer Versuchsaufbau für die Messung der Wirkleistung für allgemeine
	Verbraucher in Sternschaltung. (Bei den Kabeln wurde ein Farbschema eingehalten, um eine bessere Übersicht zu ermöglichen.) 
	1 \(\dots\) Versorgungsspannung ($L_1$ rot, $L_2$ blau, $L_3$ gelb) \\
	2 \(\dots\) seriell geschaltete Strommessgeräte  \\
	3 \(\dots\) seriell geschaltete Leistungsmessgeräte mit parallelen Anschlüssen zum Neutralleiter (grün)\\
	4 \(\dots\) parallel geschaltete analoge Spannungsmessgeräte über die entsprechenden Verbraucher (schwarz)\\
	5 \(\dots\) parallel geschaltete digitale Spannungsmessgeräte über die entsprechenden Verbraucher (schwarz/grün)\\
	6 \(\dots\) Strommessgerät zwischen Sternpunkt und Neutralleiter (grau) \\
	7 \(\dots\) Heizwiderstände \\
	8 \(\dots\) ohmscher Verbraucher \\
	9 \(\dots\) Kapazität (Kondensator) \\
	10 \(\dots\) Induktivität (Spule) \\
	11 \(\dots\) 2. Kapazität für Bonusaufgabe
	
	}\label{fig:aufbau3_echt}
\end{figure}

Für die Bonusaufgabe werden folgende Änderungen vorgenommen:

\begin{itemize}
	\item $L_1$ bleibt unverändert (Heizwiderstand)
	\item $L_2$ Schaltung von einem Heizwiderstand und einem Kondensator mit parallel geschalteter Induktivität
	\item $L_3$ Schaltung von einem Heizwiderstand und einem Kondensator
\end{itemize}


\subsection{Blindleistungsmessung}

Um die Blindleistung eines allgemeinen Verbrauchers sichtbar zu machen, wird nun die Schaltung nach folgendem Schaltplan aus 
\autoref{fig:aufbau4} aufgebaut, indem die grünen Kabel der Powermeter aus \autoref{fig:aufbau3_echt} entsprechend modifiziert werden.

\begin{figure}[H]
	\begin{center}
		\includegraphics[width = 0.8\textwidth]{./figures/aufbau4.png}
	\end{center}
	\caption[Schaltplan für die Messung der Blindleistung für allgemeine
	Verbraucher in Sternschaltung]
	{Schaltplan für die Messung der Blindleistung für allgemeine
	Verbraucher in Sternschaltung~\cite[]{leistungsmessungvorbereitung}  \\
	$I_i$ \(\dots\) entsprechende Ströme gemessen mit entsprechenden Amperemeter A  \\
	$U_i$ \(\dots\) entsprechende Spannungen gemessen mit entsprechenden Voltmeter V  \\
	$R_i$ \(\dots\) entsprechender Widerstand durch die jeweiligen Verbraucher  \\
	$P_i$ \(\dots\) Powermeter}
	\label{fig:aufbau4}
\end{figure}

\subsection{Bau eines rudimentärern Asynchron-Drehstrommotors}

Um den Bau eines rudimentären Asynchron-Drehstrommotors zu realisieren, werden 3 Spulen mit Eisenkern wie in 
\autoref{fig:motor} um eine drehbar gelagerte Metallscheibe aufgestellt. Die Spulen werden mit vorgeschalteten 
Heizwiderständen an die Versorgungsspannung geschlossen.

%todo max bitte das Bild vom Motor
%\label{fig:motor}



\section{Geräteliste}
\label{sec:geraeteliste}





\section{Versuchsdurchführung und Messergebnisse}
\label{sec:versuchsdurchfuehrung_messergebnisse}

\subsection{ohmsche Last in Wechselstromkreis}


\subsection{Symmetrische Last in Dreieckschaltung}


\subsection{Symmetrische Last in Sternschaltung}


\subsection{Asymmetrische Last in Sternschaltung}


\subsection{Asymmetrische Last in Sternschaltung und simulierten Kabelbruch}


\subsection{Wirkleistungsmessung}


\subsection{Blindleistungsmessung}


\subsection{Bau eines rudimentärern Asynchron-Drehstrommotors}


\section{Auswertung}
\label{sec:auswertung}

\subsection{ohmsche Last in Wechselstromkreis}


\subsection{Symmetrische Last in Dreieckschaltung}


\subsection{Symmetrische Last in Sternschaltung}


\subsection{Asymmetrische Last in Sternschaltung}


\subsection{Asymmetrische Last in Sternschaltung und simulierten Kabelbruch}


\subsection{Wirkleistungsmessung}


\subsection{Blindleistungsmessung}


\subsection{Bau eines rudimentärern Asynchron-Drehstrommotors}


\section{Diskussion}
\label{sec:diskussion}

\subsection{ohmsche Last in Wechselstromkreis}


\subsection{Symmetrische Last in Dreieckschaltung}


\subsection{Symmetrische Last in Sternschaltung}


\subsection{Asymmetrische Last in Sternschaltung}


\subsection{Asymmetrische Last in Sternschaltung und simulierten Kabelbruch}


\subsection{Wirkleistungsmessung}


\subsection{Blindleistungsmessung}


\subsection{Bau eines rudimentärern Asynchron-Drehstrommotors}

\section{Zusammenfassung}
\label{sec:zusammenfassung}

\subsection{ohmsche Last in Wechselstromkreis}


\subsection{Symmetrische Last in Dreieckschaltung}


\subsection{Symmetrische Last in Sternschaltung}


\subsection{Asymmetrische Last in Sternschaltung}


\subsection{Asymmetrische Last in Sternschaltung und simulierten Kabelbruch}


\subsection{Wirkleistungsmessung}


\subsection{Blindleistungsmessung}


\subsection{Bau eines rudimentärern Asynchron-Drehstrommotors}

\newpage

\printbibliography
\listoffigures
\listoftables
\end{document}